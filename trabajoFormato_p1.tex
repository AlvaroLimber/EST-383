\documentclass[11pt,openbib, letter]{article}

\oddsidemargin=1cm

% Incluir los paquetes necesarios 
\usepackage[latin1]{inputenc}%paquete principal2
\usepackage{latexsym} % Simbolos 
\usepackage{graphicx} % Inclusion de graficos. Soporte para figura 
\usepackage{hyperref} % Soporte hipertexto
\usepackage[spanish,es-noquoting]{babel}
\usepackage{subfigure}
\usepackage{epsfig}
\usepackage{graphicx}
\usepackage{rotating}
\usepackage{epstopdf}
\usepackage{ctable}
\usepackage{longtable}
\usepackage{tabularx,colortbl}
\usepackage{setspace}
\usepackage{threeparttable}
\usepackage{multirow}
\usepackage{pdflscape}
\usepackage{anysize}
%\usepackage[authoryear]{natbib}
\usepackage{breakurl}
\usepackage{url}
\usepackage{amssymb}
\usepackage{graphicx}
\usepackage[centertags]{amsmath}
\usepackage{amsthm}
\usepackage{array}
\usepackage{times}
\usepackage[left=1in, right=1in, top=1in, bottom=1in]{geometry}
\usepackage{rotating}
\usepackage{amstext}
\usepackage{pdfpages}
\usepackage{amsbsy}
\usepackage{amsopn}
\usepackage{eucal}
\usepackage{sectsty}
\usepackage{titlesec}
\usepackage[capposition=top]{floatrow}
\usepackage{pdfpages}
\usepackage{apacite}
\usepackage{tabularx,colortbl}
%\usepackage[singlespacing]{setspace}
%\usepackage[longnamesfirst,comma]{natbib}
%\usepackage{iadb}
\usepackage{authblk}
\title{{\Large \textbf{Contenido m�nimo para el Trabajo de Investigaci�n}}}
\author{Docente: Alvaro Limber Chirino Gutierrez}
\affil{Materia: Programaci�n I\\ Universidad Mayor de San Andr�s}

\date{Febrero, 2020}

\begin{document} % Inicio del documento
%\renewcommand{\abstractname}{Resumen}
%\renewcommand{\contentsname}{Contenido} 
%\renewcommand{\appendixname}{Apéndice} 
%\renewcommand{\refname}{Referencias} 
%\renewcommand{\figurename}{Figura} 
%\renewcommand{\listfigurename}{�?ndice de figuras} 
%\renewcommand{\tablename}{Tabla}
\maketitle
%%%%%%%%%%%%%%%%%%%%%%%%%%%%%%%%%%%%%%%%%%%%%%%
%%%%%%%%%%%%%%%%%%%%%%%%%%%%%%%%%%%%%%%%%%%%%%%
\section{Caracter�sticas}
\begin{itemize}
\item Tipo de trabajo: Plataforma de indicadores implementada en Shiny
\begin{itemize}
\item Debe trabajar sobre una base de datos nacional o internacional (que incluya a Bolivia)
\item La tem�tica de la base de datos es a elecci�n del estudiante
\item Se debe considerar al menos 10 indicadores
\item Bases de datos con al menos 10 mil observaciones y 10 variables
\item El estilo y contenido de la visualizaci�n de la informaci�n en la plataforma depende del estudiante
\item El c�digo debe incluir rutinas de consulta a Spark (Opcional)
\end{itemize}
\item El trabajo puede realizarse en grupos de m�ximo 2 personas
\item Los trabajos no deben exceder las 15 paginas (sin contar anexos y caratula)
\item Al final se debe preparar una presentaci�n del trabajo, (m�ximo 10 minutos)
\item (Opcional) Se valorara de forma positiva que el trabajo este en Latex
\item Entregables
\begin{itemize}
\item C�digo de R de la plataforma final (Idealmente funcional desde Github)
\item Documento de trabajo (informe)
\end{itemize}
\end{itemize}

\section{Contenido m�nimo del documento de trabajo}
El contenido m�nimo del trabajo de investigaci�n es:
\begin{itemize}
\item \textbf{Introducci�n}
\item \textbf{Objetivos}
\item \textbf{Motivaci�n}
\item \textbf{Marco Te�rico} Descripci�n de la tem�tica de estudio, la funcionalidad de R, etc.
\item \textbf{Descripci�n de la base de datos}
\item \textbf{Metodolog�a}
\begin{enumerate}
\item Indicadores y fichas de Indicadores 
\item Tratamiento sobre la base de datos
\item Visualizaci�n de la informaci�n
\item Armado de la plataforma
\end{enumerate}
\item \textbf{Resultados y an�lisis}
\item \textbf{Conclusiones y recomendaciones}
\end{itemize}

\section{Fechas Importantes}

\begin{itemize}
\item \textbf{Primera revisi�n (5 pts.):} 30 de marzo, hasta la descripci�n de la base de datos
\item \textbf{Segunda revisi�n (5 pts.):} 29 de abril, hasta la metodolog�a
\item \textbf{Informe final y presentaci�n (30 pts.):} 8, 10 de Junio
\end{itemize}

\end{document}